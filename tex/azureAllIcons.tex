
% Copyright Glen Newton glen.newton@gmail.com
%
\documentclass[12pt]{article}
\usepackage{imakeidx}
\usepackage{hyperref}
\usepackage[hyphenbreaks]{breakurl}
\usepackage[most]{tcolorbox}
%\usepackage{savetrees}
\usepackage{graphicx}
\usepackage[utf8]{inputenc}
\usepackage[english]{babel}
\usepackage{multicol}
\usepackage{seqsplit}
\usepackage{xcolor}
\usepackage{etoolbox}
\usepackage{tocloft}
\usepackage{fancyhdr}
\usepackage{listings}
\usepackage{tabularx}
\usepackage[export]{adjustbox}
\usepackage{multicol}

\usepackage[inner=10mm,outer=10mm]{geometry}


%\usepackage{../sty/azureicons} 

\graphicspath{{../icons_tex/}}

%%%%%%%%%%%%%%%%%%%%%%%%%%%
\title{\LaTeX\ Style for Azure Icons}
\author{Glen Newton \\\texttt{\href{mailto:glen.newton@gmail.com}{glen.newton@gmail.com}}}

\date{\today}

%%%%%%%%%%%%%%%%%%%%%%%%%%%%%%%%%%%%%%%%%%%%%%%%%%%%%%%%%%%%%%%%%%%%%%%%
%% Change below lines to change look of document %%%%%%%%%%%%%%%%%%%%%%%
%%%%%%%%%%%%%%%%%%%%%%%%%%%%%%%%%%%%%%%%%%%%%%%%%%%%%%%%%%%%%%%%%%%%%%%%

\pagestyle{fancy}
\fancyhf{}
\rfoot{\hyperref[index]{General Index}}
\cfoot{\hyperref[macros]{Macro Index}}
\lfoot{\hyperref[toc]{TOC}}

\rhead{Page \thepage}
\chead{\href{https://github.com/gnewton/azureArchIcons2LaTeX}{azureArchIcons2LaTeX}}

\newcommand{\iconsize}{1cm}
\newcommand{\entrySpacing}{2mm}
\newcommand{\minipagespacing}{8cm}
%\renewcommand{\familydefault}{\sfdefault}

%% Arch and Res sections start/end
\newcommand{\archStart}{\section{Services}\begin{multicols}{2}\footnotesize}
\newcommand{\archEnd}{\end{multicols}\clearpage}
\newcommand{\resStart}{\newpage\section{Resources}\begin{multicols}{2}\footnotesize}
\newcommand{\resEnd}{\end{multicols}\clearpage}

%% Command used for each icon entry
%\newcommand{\gxs}[5]{\vcenteredinclude{#2}{\hspace{3mm} \begin{minipage}{\minipagespacing}{\bf \small{#1}}\\ {\footnotesize \tt \seqsplit{#3}} \\\tt Macro:\seqsplit{#4}\index[macros]{#4 (#5)} \end{minipage}}\vspace{\vspacing}\newline}

% 1 English name
% 2 graphic
% 3 PDF file (printed TeX)
% 4 Macro name
% 5 Group
\newcommand{\gxs}[5]{
\noindent\begin{tabularx}{9cm}{p{12mm}p{7.8cm}}
               {#2}& 
               #1\index{#1 (#5)}\newline
               \texttt{#3}\newline
               Macro: {\textbackslash}#4 \index[macros]{#4 (#5)}\newline \\
\end{tabularx}
\vspace{\entrySpacing}
}

%% From: https://tex.stackexchange.com/a/847
\hypersetup{
    colorlinks,
    linkcolor={red!50!black},
    citecolor={blue!50!black},
    urlcolor={blue!80!black}
}

%%%%%%%%%%%%%%%%%%%%%%%%%%%%%%%%%%%%%%%%%%%%%%%%%%%%%%%%%%%%%%%%%%%%%%%%
%% Do not change rest of preamble (below this line) unless you know what
%% you are doing %%%%%%%%%%%%%%%%%%%%%%%%%%%%%%%%%%%%%%%%%%%%%%%%%%%%%%%
%%%%%%%%%%%%%%%%%%%%%%%%%%%%%%%%%%%%%%%%%%%%%%%%%%%%%%%%%%%%%%%%%%%%%%%%

%% From: https://codepunk.io/change-the-background-color-of-a-blockquote-in-latex-2/
\definecolor{block-gray}{gray}{.75}
\newtcolorbox{blockquote}{colback=block-gray,grow to right by=-1mm,grow to left by=-1mm,boxrule=0pt,boxsep=0pt,breakable}

\renewcommand{\cftsecleader}{\cftdotfill{\cftdotsep}}

\makeatletter
\appto{\endmulticols}{\@doendpe}
\makeatother

%% Ref to index: From https://tex.stackexchange.com/questions/182710/how-to-reference-the-index
\makeindex[intoc,title={Index: General\label{index}}]
\makeindex[name=macros, intoc,title={Index: \LaTeX \ \ Macros\label{macros}}]

%% "Too many open files" problem: From: https://stackoverflow.com/questions/1715677/error-too-many-open-files-in-pdflatex/1720556#1720556
\let\mypdfximage\pdfximage
\def\pdfximage{\immediate\mypdfximage}

 \newcommand{\vcenteredinclude}[1]{\begingroup
   \setbox0=\hbox{#1}%
   \parbox[t]{\wd0}{\box0}\endgroup}

%%%%%%%%%%%%%%%%%%%%%%%%%%%%%%%%%%%%%%%%%%%%%%%%%%%%%%%%%%%%%%%%%%%%%%%
\begin{document}

\maketitle
\tableofcontents\label{toc}
%\clearpage
\noindent\makebox[\linewidth]{\rule{\paperwidth}{0.4pt}}

%%%%%%
\section{What is this?}
\LaTeX\ style (\texttt{awsicons.sty}) which allows you to use the \href{https://aws.amazon.com/architecture/icons/}{AWS architectural icons} in your \LaTeX\  documents.

\vspace{3mm}

%\noindent The AWS SVG icons are converted to {\LaTeX}--compatible PDF form using \href{https://inkscape.org/}{Inkscape} \texttt{\inkscapeVersion} using the purpose--written Go program \href{https://github.com/gnewton/awsArchIcons2LaTeX/aws2tex}{aws2tex}. 

\noindent This Go program, \texttt{aws2tex}, creates:
\begin{itemize}
\item \texttt{awsicons.sty} file
\item \LaTeX--compatible PDF versions of icons (and written to the \texttt{icons\_tex} directory, using Inkscape).
\item This document, which is to illustrate as well as index these icons and their corresponding \LaTeX\ \  macros.
  \newline
  \textbf{NB:} Not all icons are converted:
  \begin{itemize}
  \item For \textit{Resources}, only the \texttt{\_Light} versions. The \texttt{\_Dark} are omitted.
  \item For \textit{Services}, only the \texttt{\_64} versions. The lower resolution \texttt{\_16,\_32,\_48} are omitted.
  \end{itemize}
For instructions on how to generate this document with \textit{all} icons, please see the instructions in the \href{https://github.com/gnewton/awsArchIcons2LaTeX}{README.md} at the github repo.\\
This document is: \url{https://github.com/gnewton/awsArchIcons2LaTeX/awsAllIcons.pdf}
\end{itemize}
This document also includes hyperlinks to Google searches of the titles of each icon.


%%%%%%
\section{How to use it?}
There are two parts:
\begin{enumerate}
\item \texttt{awsicons.sty} file
\item directory with all of the icons rendered into \LaTeX\ compatible form (in the repo, this is: \texttt{icons\_tex}).
\end{enumerate}
All you need to do is use the package to your \LaTeX\ document using \texttt{{\textbackslash}usepackage} (after having added the \texttt{awsicons.sty} file to where \LaTeX\ can see it, or put it in the same directory as your document) AND add the directory containing all the generated icons to the \texttt{{\textbackslash}graphicspath\{...\}}.\\

\noindent For example:
\vspace{4mm}

\begin{blockquote}
\begin{verbatim}
% preamble
\usepackage{awsicons} 
\graphicspath{../icons_tex/}
% after \begin{document}
Here is a 1cm API Gateway:
\RAPIGatewayEndpoint{1cm}
Here is a 3cm Lambda function:
\RLambdaLambdaFunc{3cm}

\end{verbatim}
\end{blockquote}

\noindent Resulting in:\\
\begin{blockquote}
Here is a 1cm API Gateway:
%\RAPIGatewayEndpoint{1cm}
Here is a 3cm Lambda function:
%\RLambdaLambdaFunc{3cm}

\end{blockquote}

[Without the gray background]
    
%%%%%%
\subsection{Examples}
There are more complex examples in the \href{https://github.com/gnewton/awsArchIcons2LaTeX/tree/main/examples}{examples} directory, including one which recreates an AWS Architecture blog diagram.

\begin{itemize}
\item \href{https://github.com/gnewton/awsArchIcons2LaTeX/blob/main/examples/simple.pdf}{Simple example}
\item AWS Architecture blog diagram: \textit{\href{https://github.com/gnewton/awsArchIcons2LaTeX/blob/main/examples/Data-pipeline-Grov-Technologies.pdf}{Building a Controlled Environment Agriculture Platform}\footnote{\url{https://aws.amazon.com/blogs/architecture/building-a-controlled-environment-agriculture-platform/}}}, by Ashu Joshi, 2020.12.22.
\end{itemize}
  
\section{Sources}
The AWS Architecture icons used here are from \url{https://aws.amazon.com/architecture/icons/}.\\
%
%The specific assets package used: \href{https://d1.awsstatic.com/webteam/architecture-icons/Q32020/\assetZipFile}{\assetZipFileSplit}
%The specific assets package used: {\sloppy \href{https://d1.awsstatic.com/webteam/architecture-icons/Q32020/}{\assetZipFile}}

\vspace{3cm}
\noindent \copyright\ Glen Newton 2020\\
Icons in this document (I believe) are: \ \copyright\ Amazon Web Services (AWS)
\vspace{5mm}

\setlength{\parindent}{0pt}

\begin{multicols}{2}\footnotesize
%%%%%%%%%%%%% ICON CONTENT: generated by azure2tex:
\subsection{AI + Machine Learning}
\gxs{Bot Services}{\BotSvcs{1cm}}{10165\_BotServices.pdf}{BotSvcs}{AI + Machine Learning}

\gxs{Cognitive Services}{\CognitiveSvcs{1cm}}{10162\_CognitiveServices.pdf}{CognitiveSvcs}{AI + Machine Learning}

\subsection{Analytics}
\gxs{Analysis Services}{\AnalysisSvcs{1cm}}{10148\_AnalysisServices.pdf}{AnalysisSvcs}{Analytics}

\gxs{Azure Synapse Analytics}{\AzrSynapseAnalytics{1cm}}{00606\_AzureSynapseAnalytics.pdf}{AzrSynapseAnalytics}{Analytics}

\gxs{Data Lake Store Gen1}{\DataLakeStoreGenOne{1cm}}{10150\_DataLakeStoreGen1.pdf}{DataLakeStoreGenOne}{Analytics}

\gxs{Event Hub Clusters}{\EventHubClusters{1cm}}{10149\_EventHubClusters.pdf}{EventHubClusters}{Analytics}

\gxs{Event Hubs}{\EventHubs{1cm}}{00039\_EventHubs.pdf}{EventHubs}{Analytics}

\gxs{HD Insight Clusters}{\HDInsightClusters{1cm}}{10142\_HDInsightClusters.pdf}{HDInsightClusters}{Analytics}

\gxs{Log Analytics Workspaces}{\LogAnalyticsWorkspaces{1cm}}{00009\_LogAnalyticsWorkspaces.pdf}{LogAnalyticsWorkspaces}{Analytics}

\gxs{Stream Analytics Jobs}{\StrmAnalyticsJobs{1cm}}{00042\_StreamAnalyticsJobs.pdf}{StrmAnalyticsJobs}{Analytics}

\subsection{App Services}
\gxs{API Management Services}{\APIMngmtSvcs{1cm}}{10042\_APIManagementServices.pdf}{APIMngmtSvcs}{App Services}

\gxs{App Service Certificates}{\AppSvcCerts{1cm}}{00049\_AppServiceCertificates.pdf}{AppSvcCerts}{App Services}

\gxs{App Service Domains}{\AppSvcDomains{1cm}}{00050\_AppServiceDomains.pdf}{AppSvcDomains}{App Services}

\gxs{App Service Environments}{\AppSvcEnvironments{1cm}}{10047\_AppServiceEnvironments.pdf}{AppSvcEnvironments}{App Services}

\gxs{App Service Plans}{\AppSvcPlans{1cm}}{00046\_AppServicePlans.pdf}{AppSvcPlans}{App Services}

\gxs{App Services}{\AppSvcs{1cm}}{10035\_AppServices.pdf}{AppSvcs}{App Services}

\gxs{CDN Profiles}{\CDNProfs{1cm}}{00056\_CDNProfiles.pdf}{CDNProfs}{App Services}

\gxs{Notification Hubs}{\NotifHubs{1cm}}{10045\_NotificationHubs.pdf}{NotifHubs}{App Services}

\gxs{Search Services}{\SrchSvcs{1cm}}{10044\_SearchServices.pdf}{SrchSvcs}{App Services}

\subsection{Azure Stack}
\gxs{Azure Stack}{\AzrStack{1cm}}{10114\_AzureStack.pdf}{AzrStack}{Azure Stack}

\gxs{Capacity}{\Capacity{1cm}}{10109\_Capacity.pdf}{Capacity}{Azure Stack}

\gxs{Infrastructure Backup}{\InfrastructureBackup{1cm}}{10108\_InfrastructureBackup.pdf}{InfrastructureBackup}{Azure Stack}

\gxs{Multi Tenancy}{\MultiTenancy{1cm}}{00965\_MultiTenancy.pdf}{MultiTenancy}{Azure Stack}

\gxs{Offers}{\Offers{1cm}}{10110\_Offers.pdf}{Offers}{Azure Stack}

\gxs{Plans}{\Plans{1cm}}{10113\_Plans.pdf}{Plans}{Azure Stack}

\gxs{Updates}{\Updates{1cm}}{10115\_Updates.pdf}{Updates}{Azure Stack}

\gxs{User Subscriptions}{\UserSubscriptions{1cm}}{10111\_UserSubscriptions.pdf}{UserSubscriptions}{Azure Stack}

\subsection{Azure VMware Solution}
\gxs{AVS}{\AVS{1cm}}{00524\_AVS.pdf}{AVS}{Azure VMware Solution}

\subsection{Blockchain}
\gxs{ABS Member}{\ABSMember{1cm}}{10374\_ABSMember.pdf}{ABSMember}{Blockchain}

\gxs{Azure Blockchain Service}{\AzrBlockchainSvc{1cm}}{10366\_AzureBlockchainService.pdf}{AzrBlockchainSvc}{Blockchain}

\gxs{Azure Token Service}{\AzrTokenSvc{1cm}}{10367\_AzureTokenService.pdf}{AzrTokenSvc}{Blockchain}

\gxs{Consortium}{\Consortium{1cm}}{10375\_Consortium.pdf}{Consortium}{Blockchain}

\gxs{Outbound Connection}{\OutboundConnection{1cm}}{10364\_OutboundConnection.pdf}{OutboundConnection}{Blockchain}

\subsection{Compute}
\gxs{App Services}{\AppSvcs{1cm}}{10035\_AppServices.pdf}{AppSvcs}{Compute}

\gxs{Availability Sets}{\AvailabilitySets{1cm}}{10025\_AvailabilitySets.pdf}{AvailabilitySets}{Compute}

\gxs{Batch Accounts}{\BatchAccnts{1cm}}{10031\_BatchAccounts.pdf}{BatchAccnts}{Compute}

\gxs{Cloud Services Classic}{\CloudSvcsClassic{1cm}}{10030\_CloudServicesClassic.pdf}{CloudSvcsClassic}{Compute}

\gxs{Container Instances}{\ContainerInsts{1cm}}{10104\_ContainerInstances.pdf}{ContainerInsts}{Compute}

\gxs{Container Services Deprecated}{\ContainerSvcsDeprecated{1cm}}{10049\_ContainerServicesDeprecated.pdf}{ContainerSvcsDeprecated}{Compute}

\gxs{Disk Encryption Sets}{\DiskEncrSets{1cm}}{00398\_DiskEncryptionSets.pdf}{DiskEncrSets}{Compute}

\gxs{Disks}{\Disks{1cm}}{10032\_Disks.pdf}{Disks}{Compute}

\gxs{Disks Classic}{\DisksClassic{1cm}}{10041\_DisksClassic.pdf}{DisksClassic}{Compute}

\gxs{Disks Snapshots}{\DisksSnapshots{1cm}}{10026\_DisksSnapshots.pdf}{DisksSnapshots}{Compute}

\gxs{Function Apps}{\FuncApps{1cm}}{10029\_FunctionApps.pdf}{FuncApps}{Compute}

\gxs{Image Definitions}{\ImgDefinitions{1cm}}{10037\_ImageDefinitions.pdf}{ImgDefinitions}{Compute}

\gxs{Image Versions}{\ImgVersions{1cm}}{10038\_ImageVersions.pdf}{ImgVersions}{Compute}

\gxs{Images}{\Imgs{1cm}}{10033\_Images.pdf}{Imgs}{Compute}

\gxs{Kubernetes Services}{\KuberSvcs{1cm}}{10023\_KubernetesServices.pdf}{KuberSvcs}{Compute}

\gxs{Mesh Applications}{\MeshApps{1cm}}{10024\_MeshApplications.pdf}{MeshApps}{Compute}

\gxs{OS Images Classic}{\OSImgsClassic{1cm}}{10027\_OSImagesClassic.pdf}{OSImgsClassic}{Compute}

\gxs{Service Fabric Clusters}{\SvcFabricClusters{1cm}}{10036\_ServiceFabricClusters.pdf}{SvcFabricClusters}{Compute}

\gxs{Shared Image Galleries}{\SharedImgGalleries{1cm}}{10039\_SharedImageGalleries.pdf}{SharedImgGalleries}{Compute}

\gxs{VM Images Classic}{\VMImgsClassic{1cm}}{10040\_VMImagesClassic.pdf}{VMImgsClassic}{Compute}

\gxs{VM Scale Sets}{\VMScaleSets{1cm}}{10034\_VMScaleSets.pdf}{VMScaleSets}{Compute}

\gxs{Virtual Machine}{\VirtMachine{1cm}}{10021\_VirtualMachine.pdf}{VirtMachine}{Compute}

\gxs{Virtual Machines Classic}{\VirtMachinesClassic{1cm}}{10028\_VirtualMachinesClassic.pdf}{VirtMachinesClassic}{Compute}

\gxs{Workspaces}{\Workspaces{1cm}}{00400\_Workspaces.pdf}{Workspaces}{Compute}

\subsection{Containers}
\gxs{App Services}{\AppSvcs{1cm}}{10035\_AppServices.pdf}{AppSvcs}{Containers}

\gxs{Batch Accounts}{\BatchAccnts{1cm}}{10031\_BatchAccounts.pdf}{BatchAccnts}{Containers}

\gxs{Container Instances}{\ContainerInsts{1cm}}{10104\_ContainerInstances.pdf}{ContainerInsts}{Containers}

\gxs{Container Registries}{\ContainerRegistries{1cm}}{10105\_ContainerRegistries.pdf}{ContainerRegistries}{Containers}

\gxs{Kubernetes Services}{\KuberSvcs{1cm}}{10023\_KubernetesServices.pdf}{KuberSvcs}{Containers}

\gxs{Service Fabric Clusters}{\SvcFabricClusters{1cm}}{10036\_ServiceFabricClusters.pdf}{SvcFabricClusters}{Containers}

\subsection{Databases}
\gxs{Azure Cosmos DB}{\AzrCosmosDB{1cm}}{10121\_AzureCosmosDB.pdf}{AzrCosmosDB}{Databases}

\gxs{Azure Data Explorer Clusters}{\AzrDataExplorerClusters{1cm}}{10145\_AzureDataExplorerClusters.pdf}{AzrDataExplorerClusters}{Databases}

\gxs{Azure Database MariaDB Server}{\AzrDBMariaDBServer{1cm}}{10123\_AzureDatabaseMariaDBServer.pdf}{AzrDBMariaDBServer}{Databases}

\gxs{Azure Database Migration Services}{\AzrDBMigratSvcs{1cm}}{10133\_AzureDatabaseMigrationServices.pdf}{AzrDBMigratSvcs}{Databases}

\gxs{Azure Database MySQL Server}{\AzrDBMySQLServer{1cm}}{10122\_AzureDatabaseMySQLServer.pdf}{AzrDBMySQLServer}{Databases}

\gxs{Azure Database PostgreSQL Server}{\AzrDBPostgreSQLServer{1cm}}{10131\_AzureDatabasePostgreSQLServer.pdf}{AzrDBPostgreSQLServer}{Databases}

\gxs{Azure SQL}{\AzrSQL{1cm}}{02390\_AzureSQL.pdf}{AzrSQL}{Databases}

\gxs{Azure SQL Server Stretch Databases}{\AzrSQLServerStretchDBs{1cm}}{10137\_AzureSQLServerStretchDatabases.pdf}{AzrSQLServerStretchDBs}{Databases}

\gxs{Azure SQL VM}{\AzrSQLVM{1cm}}{10124\_AzureSQLVM.pdf}{AzrSQLVM}{Databases}

\gxs{Azure Synapse Analytics}{\AzrSynapseAnalytics{1cm}}{00606\_AzureSynapseAnalytics.pdf}{AzrSynapseAnalytics}{Databases}

\gxs{Cache Redis}{\CacheRedis{1cm}}{10137\_CacheRedis.pdf}{CacheRedis}{Databases}

\gxs{Data Factory}{\DataFactory{1cm}}{10126\_DataFactory.pdf}{DataFactory}{Databases}

\gxs{Elastic Job Agents}{\ElasticJobAgents{1cm}}{10128\_ElasticJobAgents.pdf}{ElasticJobAgents}{Databases}

\gxs{Instance Pools}{\InstPools{1cm}}{10139\_InstancePools.pdf}{InstPools}{Databases}

\gxs{Managed Database}{\MngdDB{1cm}}{10135\_ManagedDatabase.pdf}{MngdDB}{Databases}

\gxs{SQL Data Warehouses}{\SQLDataWarehouses{1cm}}{00036\_SQLDataWarehouses.pdf}{SQLDataWarehouses}{Databases}

\gxs{SQL Database}{\SQLDB{1cm}}{10130\_SQLDatabase.pdf}{SQLDB}{Databases}

\gxs{SQL Elastic Pools}{\SQLElasticPools{1cm}}{10134\_SQLElasticPools.pdf}{SQLElasticPools}{Databases}

\gxs{SQL Managed Instance}{\SQLMngdInst{1cm}}{10136\_SQLManagedInstance.pdf}{SQLMngdInst}{Databases}

\gxs{SQL Server}{\SQLServer{1cm}}{10132\_SQLServer.pdf}{SQLServer}{Databases}

\gxs{SSIS Lift And Shift IR}{\SSISLiftAndShiftIR{1cm}}{02392\_SSISLiftAndShiftIR.pdf}{SSISLiftAndShiftIR}{Databases}

\gxs{Virtual Clusters}{\VirtClusters{1cm}}{10127\_VirtualClusters.pdf}{VirtClusters}{Databases}

\subsection{DevOps}
\gxs{Application Insights}{\AppInsights{1cm}}{00012\_ApplicationInsights.pdf}{AppInsights}{DevOps}

\gxs{Azure DevOps}{\AzrDevOps{1cm}}{10261\_AzureDevOps.pdf}{AzrDevOps}{DevOps}

\gxs{DevTest Labs}{\DevTestLabs{1cm}}{10264\_DevTestLabs.pdf}{DevTestLabs}{DevOps}

\gxs{Lab Services}{\LabSvcs{1cm}}{10265\_LabServices.pdf}{LabSvcs}{DevOps}

\subsection{General}
\gxs{All Resources}{\AllRess{1cm}}{10001\_AllResources.pdf}{AllRess}{General}

\gxs{Backlog}{\Backlog{1cm}}{10853\_Backlog.pdf}{Backlog}{General}

\gxs{Biz Talk}{\BizTalk{1cm}}{10779\_BizTalk.pdf}{BizTalk}{General}

\gxs{Blob Block}{\BlobBlock{1cm}}{10780\_BlobBlock.pdf}{BlobBlock}{General}

\gxs{Blob Page}{\BlobPg{1cm}}{10781\_BlobPage.pdf}{BlobPg}{General}

\gxs{Branch}{\Branch{1cm}}{10782\_Branch.pdf}{Branch}{General}

\gxs{Browser}{\Browser{1cm}}{10783\_Browser.pdf}{Browser}{General}

\gxs{Bug}{\Bug{1cm}}{10784\_Bug.pdf}{Bug}{General}

\gxs{Builds}{\Builds{1cm}}{10785\_Builds.pdf}{Builds}{General}

\gxs{Cache}{\Cache{1cm}}{10786\_Cache.pdf}{Cache}{General}

\gxs{Code}{\Code{1cm}}{10787\_Code.pdf}{Code}{General}

\gxs{Commit}{\Commit{1cm}}{10788\_Commit.pdf}{Commit}{General}

\gxs{Controls}{\Controls{1cm}}{10789\_Controls.pdf}{Controls}{General}

\gxs{Controls Horizontal}{\ControlsHorizontal{1cm}}{10790\_ControlsHorizontal.pdf}{ControlsHorizontal}{General}

\gxs{Cost Alerts}{\CostAlerts{1cm}}{10791\_CostAlerts.pdf}{CostAlerts}{General}

\gxs{Cost Analysis}{\CostAnalysis{1cm}}{10792\_CostAnalysis.pdf}{CostAnalysis}{General}

\gxs{Cost Budgets}{\CostBudgets{1cm}}{10793\_CostBudgets.pdf}{CostBudgets}{General}

\gxs{Cost Management}{\CostMngmt{1cm}}{10019\_CostManagement.pdf}{CostMngmt}{General}

\gxs{Counter}{\Counter{1cm}}{10794\_Counter.pdf}{Counter}{General}

\gxs{Cubes}{\Cubes{1cm}}{10795\_Cubes.pdf}{Cubes}{General}

\gxs{Dashboard}{\Dashboard{1cm}}{10015\_Dashboard.pdf}{Dashboard}{General}

\gxs{Dev Console}{\DevConsole{1cm}}{10796\_DevConsole.pdf}{DevConsole}{General}

\gxs{Download}{\Download{1cm}}{10797\_Download.pdf}{Download}{General}

\gxs{Error}{\Error{1cm}}{10798\_Error.pdf}{Error}{General}

\gxs{Extensions}{\Extensions{1cm}}{10799\_Extensions.pdf}{Extensions}{General}

\gxs{FTP}{\FTP{1cm}}{10804\_FTP.pdf}{FTP}{General}

\gxs{File}{\File{1cm}}{10800\_File.pdf}{File}{General}

\gxs{Files}{\Files{1cm}}{10801\_Files.pdf}{Files}{General}

\gxs{Folder Blank}{\FolderBlank{1cm}}{10802\_FolderBlank.pdf}{FolderBlank}{General}

\gxs{Folder Website}{\FolderWebsite{1cm}}{10803\_FolderWebsite.pdf}{FolderWebsite}{General}

\gxs{Gear}{\Gear{1cm}}{10805\_Gear.pdf}{Gear}{General}

\gxs{Globe}{\Globe{1cm}}{10806\_Globe.pdf}{Globe}{General}

\gxs{Globe Error}{\GlobeError{1cm}}{10807\_GlobeError.pdf}{GlobeError}{General}

\gxs{Globe Success}{\GlobeSuccess{1cm}}{10808\_GlobeSuccess.pdf}{GlobeSuccess}{General}

\gxs{Globe Warning}{\GlobeWarn{1cm}}{10809\_GlobeWarning.pdf}{GlobeWarn}{General}

\gxs{Guide}{\Guide{1cm}}{10810\_Guide.pdf}{Guide}{General}

\gxs{Heart}{\Heart{1cm}}{10811\_Heart.pdf}{Heart}{General}

\gxs{Help and Support}{\HelpandSupport{1cm}}{10013\_HelpandSupport.pdf}{HelpandSupport}{General}

\gxs{Image}{\Img{1cm}}{10812\_Image.pdf}{Img}{General}

\gxs{Information}{\Information{1cm}}{10005\_Information.pdf}{Information}{General}

\gxs{Input Output}{\InputOutput{1cm}}{10813\_InputOutput.pdf}{InputOutput}{General}

\gxs{Journey Hub}{\JourneyHub{1cm}}{10814\_JourneyHub.pdf}{JourneyHub}{General}

\gxs{Launch Portal}{\LaunchPortal{1cm}}{10815\_LaunchPortal.pdf}{LaunchPortal}{General}

\gxs{Learn}{\Learn{1cm}}{10816\_Learn.pdf}{Learn}{General}

\gxs{Load Test}{\LoadTest{1cm}}{10817\_LoadTest.pdf}{LoadTest}{General}

\gxs{Location}{\Location{1cm}}{10818\_Location.pdf}{Location}{General}

\gxs{Log Streaming}{\LogStrming{1cm}}{10819\_LogStreaming.pdf}{LogStrming}{General}

\gxs{Management Groups}{\MngmtGrp{1cm}}{10011\_ManagementGroups.pdf}{MngmtGrp}{General}

\gxs{Management Portal}{\MngmtPortal{1cm}}{10820\_ManagementPortal.pdf}{MngmtPortal}{General}

\gxs{Marketplace}{\Marketplace{1cm}}{10008\_Marketplace.pdf}{Marketplace}{General}

\gxs{Media}{\Media{1cm}}{10854\_Media.pdf}{Media}{General}

\gxs{Media File}{\MediaFile{1cm}}{10821\_MediaFile.pdf}{MediaFile}{General}

\gxs{Mobile}{\Mob{1cm}}{10822\_Mobile.pdf}{Mob}{General}

\gxs{Mobile Engagement}{\MobEngagement{1cm}}{10823\_MobileEngagement.pdf}{MobEngagement}{General}

\gxs{Module}{\Module{1cm}}{10855\_Module.pdf}{Module}{General}

\gxs{Power}{\Power{1cm}}{10824\_Power.pdf}{Power}{General}

\gxs{Power Up}{\PowerUp{1cm}}{10826\_PowerUp.pdf}{PowerUp}{General}

\gxs{Powershell}{\Powershell{1cm}}{10825\_Powershell.pdf}{Powershell}{General}

\gxs{Preview}{\Preview{1cm}}{10827\_Preview.pdf}{Preview}{General}

\gxs{Process Explorer}{\ProcessExplorer{1cm}}{10828\_ProcessExplorer.pdf}{ProcessExplorer}{General}

\gxs{Production Ready Database}{\ProductionReadyDB{1cm}}{10829\_ProductionReadyDatabase.pdf}{ProductionReadyDB}{General}

\gxs{Quickstart Center}{\QuickstartCenter{1cm}}{10010\_QuickstartCenter.pdf}{QuickstartCenter}{General}

\gxs{Recent}{\Recent{1cm}}{10006\_Recent.pdf}{Recent}{General}

\gxs{Reservations}{\Reservations{1cm}}{10003\_Reservations.pdf}{Reservations}{General}

\gxs{Resource Group List}{\ResGroupList{1cm}}{10830\_ResourceGroupList.pdf}{ResGroupList}{General}

\gxs{Resource Groups}{\ResGrp{1cm}}{10007\_ResourceGroups.pdf}{ResGrp}{General}

\gxs{Resource Linked}{\ResLinked{1cm}}{10831\_ResourceLinked.pdf}{ResLinked}{General}

\gxs{SSD}{\SSD{1cm}}{10837\_SSD.pdf}{SSD}{General}

\gxs{Scale}{\Scale{1cm}}{10832\_Scale.pdf}{Scale}{General}

\gxs{Scheduler}{\Scheduler{1cm}}{10833\_Scheduler.pdf}{Scheduler}{General}

\gxs{Search}{\Srch{1cm}}{10834\_Search.pdf}{Srch}{General}

\gxs{Search Grid}{\SrchGrid{1cm}}{10856\_SearchGrid.pdf}{SrchGrid}{General}

\gxs{Server Farm}{\ServerFarm{1cm}}{10835\_ServerFarm.pdf}{ServerFarm}{General}

\gxs{Service Bus}{\SvcBus{1cm}}{10836\_ServiceBus.pdf}{SvcBus}{General}

\gxs{Service Health}{\SvcHealth{1cm}}{10004\_ServiceHealth.pdf}{SvcHealth}{General}

\gxs{Storage Azure Files}{\StorAzrFiles{1cm}}{10838\_StorageAzureFiles.pdf}{StorAzrFiles}{General}

\gxs{Storage Container}{\StorContainer{1cm}}{10839\_StorageContainer.pdf}{StorContainer}{General}

\gxs{Storage Queue}{\StorQueue{1cm}}{10840\_StorageQueue.pdf}{StorQueue}{General}

\gxs{Subscriptions}{\Subscriptions{1cm}}{10002\_Subscriptions.pdf}{Subscriptions}{General}

\gxs{TFS VC Repository}{\TFSVCRepository{1cm}}{10843\_TFSVCRepository.pdf}{TFSVCRepository}{General}

\gxs{Table}{\Table{1cm}}{10841\_Table.pdf}{Table}{General}

\gxs{Tags}{\Tags{1cm}}{10842\_Tags.pdf}{Tags}{General}

\gxs{Toolbox}{\Toolbox{1cm}}{10844\_Toolbox.pdf}{Toolbox}{General}

\gxs{Versions}{\Versions{1cm}}{10845\_Versions.pdf}{Versions}{General}

\gxs{Web Slots}{\WebSlots{1cm}}{10849\_WebSlots.pdf}{WebSlots}{General}

\gxs{Web Test}{\WebTest{1cm}}{10850\_WebTest.pdf}{WebTest}{General}

\gxs{Website Power}{\WebsitePower{1cm}}{10847\_WebsitePower.pdf}{WebsitePower}{General}

\gxs{Website Staging}{\WebsiteStaging{1cm}}{10848\_WebsiteStaging.pdf}{WebsiteStaging}{General}

\gxs{Workbooks}{\Workbooks{1cm}}{10851\_Workbooks.pdf}{Workbooks}{General}

\gxs{Workflow}{\Workflow{1cm}}{10852\_Workflow.pdf}{Workflow}{General}

\subsection{Identity}
\gxs{Active Directory Connect Health}{\ActiveDirConnectHealth{1cm}}{10224\_ActiveDirectoryConnectHealth.pdf}{ActiveDirConnectHealth}{Identity}

\gxs{App Registrations}{\AppRegistrations{1cm}}{10232\_AppRegistrations.pdf}{AppRegistrations}{Identity}

\gxs{Azure AD B2C}{\AzrADBTwoC{1cm}}{10228\_AzureADB2C.pdf}{AzrADBTwoC}{Identity}

\gxs{Azure AD Domain Services}{\AzrADDomainSvcs{1cm}}{10222\_AzureADDomainServices.pdf}{AzrADDomainSvcs}{Identity}

\gxs{Azure Active Directory}{\AzrActiveDir{1cm}}{10221\_AzureActiveDirectory.pdf}{AzrActiveDir}{Identity}

\gxs{Enterprise Applications}{\EntrprsApps{1cm}}{10225\_EnterpriseApplications.pdf}{EntrprsApps}{Identity}

\gxs{Groups}{\Grp{1cm}}{10223\_Groups.pdf}{Grp}{Identity}

\gxs{Identity Governance}{\IdentGovern{1cm}}{10235\_IdentityGovernance.pdf}{IdentGovern}{Identity}

\gxs{Users}{\Users{1cm}}{10230\_Users.pdf}{Users}{Identity}

\subsection{Integration}
\gxs{API Management Services}{\APIMngmtSvcs{1cm}}{10042\_APIManagementServices.pdf}{APIMngmtSvcs}{Integration}

\gxs{Azure Data Catalog}{\AzrDataCatalog{1cm}}{10216\_AzureDataCatalog.pdf}{AzrDataCatalog}{Integration}

\gxs{Event Grid Domains}{\EventGridDomains{1cm}}{10215\_EventGridDomains.pdf}{EventGridDomains}{Integration}

\gxs{Event Grid Subscriptions}{\EventGridSubscriptions{1cm}}{10221\_EventGridSubscriptions.pdf}{EventGridSubscriptions}{Integration}

\gxs{Event Grid Topics}{\EventGridTopics{1cm}}{10206\_EventGridTopics.pdf}{EventGridTopics}{Integration}

\gxs{Integration Accounts}{\IntegAccnts{1cm}}{10218\_IntegrationAccounts.pdf}{IntegAccnts}{Integration}

\gxs{Logic Apps}{\LogicApps{1cm}}{10201\_LogicApps.pdf}{LogicApps}{Integration}

\gxs{Relays}{\Relays{1cm}}{10209\_Relays.pdf}{Relays}{Integration}

\gxs{SQL Data Warehouses}{\SQLDataWarehouses{1cm}}{00036\_SQLDataWarehouses.pdf}{SQLDataWarehouses}{Integration}

\gxs{Software as a Service}{\SoftwareasaSvc{1cm}}{10213\_SoftwareasaService.pdf}{SoftwareasaSvc}{Integration}

\subsection{Internet of Things}
\gxs{Digital Twins}{\DigitalTwins{1cm}}{01030\_DigitalTwins.pdf}{DigitalTwins}{Internet of Things}

\subsection{Intune}
\gxs{Azure AD Roles and Administrators}{\AzrADRolesandAdministrators{1cm}}{10340\_AzureADRolesandAdministrators.pdf}{AzrADRolesandAdministrators}{Intune}

\gxs{Device Security Apple}{\DevSecApple{1cm}}{00399\_DeviceSecurityApple.pdf}{DevSecApple}{Intune}

\gxs{Device Security Google}{\DevSecGoogle{1cm}}{00399\_DeviceSecurityGoogle.pdf}{DevSecGoogle}{Intune}

\gxs{Device Security Windows}{\DevSecWin{1cm}}{00399\_DeviceSecurityWindows.pdf}{DevSecWin}{Intune}

\subsection{IoT}
\gxs{Azure Maps Accounts}{\AzrMapsAccnts{1cm}}{10185\_AzureMapsAccounts.pdf}{AzrMapsAccnts}{IoT}

\gxs{Device Provisioning Services}{\DevProvisioningSvcs{1cm}}{10369\_DeviceProvisioningServices.pdf}{DevProvisioningSvcs}{IoT}

\gxs{Event Hubs}{\EventHubs{1cm}}{00039\_EventHubs.pdf}{EventHubs}{IoT}

\gxs{Function Apps}{\FuncApps{1cm}}{10029\_FunctionApps.pdf}{FuncApps}{IoT}

\gxs{IoT Central Applications}{\IoTCentralApps{1cm}}{10184\_IoTCentralApplications.pdf}{IoTCentralApps}{IoT}

\gxs{IoT Hub}{\IoTHub{1cm}}{10182\_IoTHub.pdf}{IoTHub}{IoT}

\gxs{Logic Apps}{\LogicApps{1cm}}{10201\_LogicApps.pdf}{LogicApps}{IoT}

\gxs{Notification Hubs}{\NotifHubs{1cm}}{10045\_NotificationHubs.pdf}{NotifHubs}{IoT}

\gxs{Stream Analytics Jobs}{\StrmAnalyticsJobs{1cm}}{00042\_StreamAnalyticsJobs.pdf}{StrmAnalyticsJobs}{IoT}

\gxs{Time Series Insights Environments}{\TSInsightsEnvironments{1cm}}{10181\_TimeSeriesInsightsEnvironments.pdf}{TSInsightsEnvironments}{IoT}

\gxs{Time Series Insights Event Sources}{\TSInsightsEventSrcs{1cm}}{10188\_TimeSeriesInsightsEventSources.pdf}{TSInsightsEventSrcs}{IoT}

\subsection{Management + Governance}
\gxs{Activity Log}{\ActivityLog{1cm}}{00007\_ActivityLog.pdf}{ActivityLog}{Management + Governance}

\gxs{Advisor}{\Advisor{1cm}}{00003\_Advisor.pdf}{Advisor}{Management + Governance}

\gxs{Alerts}{\Alerts{1cm}}{00002\_Alerts.pdf}{Alerts}{Management + Governance}

\gxs{Application Insights}{\AppInsights{1cm}}{00012\_ApplicationInsights.pdf}{AppInsights}{Management + Governance}

\gxs{Automation Accounts}{\AutomationAccnts{1cm}}{00022\_AutomationAccounts.pdf}{AutomationAccnts}{Management + Governance}

\gxs{Azure Arc}{\AzrArc{1cm}}{00756\_AzureArc.pdf}{AzrArc}{Management + Governance}

\gxs{Azure Lighthouse}{\AzrLighthouse{1cm}}{00471\_AzureLighthouse.pdf}{AzrLighthouse}{Management + Governance}

\gxs{Blueprints}{\Blueprints{1cm}}{00006\_Blueprints.pdf}{Blueprints}{Management + Governance}

\gxs{Compliance}{\Compliance{1cm}}{00011\_Compliance.pdf}{Compliance}{Management + Governance}

\gxs{Diagnostics Settings}{\DiagnosticsSettings{1cm}}{00008\_DiagnosticsSettings.pdf}{DiagnosticsSettings}{Management + Governance}

\gxs{Education}{\Education{1cm}}{00026\_Education.pdf}{Education}{Management + Governance}

\gxs{Log Analytics Workspaces}{\LogAnalyticsWorkspaces{1cm}}{00009\_LogAnalyticsWorkspaces.pdf}{LogAnalyticsWorkspaces}{Management + Governance}

\gxs{MachinesAzureArc}{\MachinesAzrArc{1cm}}{10450\_MachinesAzureArc.pdf}{MachinesAzrArc}{Management + Governance}

\gxs{Managed Applications Center}{\MngdAppsCenter{1cm}}{10313\_ManagedApplicationsCenter.pdf}{MngdAppsCenter}{Management + Governance}

\gxs{Metrics}{\Metrics{1cm}}{00020\_Metrics.pdf}{Metrics}{Management + Governance}

\gxs{Monitor}{\Monitor{1cm}}{00001\_Monitor.pdf}{Monitor}{Management + Governance}

\gxs{My Customers}{\MyCustomers{1cm}}{00014\_MyCustomers.pdf}{MyCustomers}{Management + Governance}

\gxs{Operation Log Classic}{\OperationLogClassic{1cm}}{00024\_OperationLogClassic.pdf}{OperationLogClassic}{Management + Governance}

\gxs{Policy}{\Policy{1cm}}{10316\_Policy.pdf}{Policy}{Management + Governance}

\gxs{Recovery Services Vaults}{\RecovSvcsVlts{1cm}}{00017\_RecoveryServicesVaults.pdf}{RecovSvcsVlts}{Management + Governance}

\gxs{Resource Graph Explorer}{\ResGraphExplorer{1cm}}{10318\_ResourceGraphExplorer.pdf}{ResGraphExplorer}{Management + Governance}

\gxs{Service Providers}{\SvcPrvdr{1cm}}{00025\_ServiceProviders.pdf}{SvcPrvdr}{Management + Governance}

\gxs{Solutions}{\Solutions{1cm}}{00021\_Solutions.pdf}{Solutions}{Management + Governance}

\gxs{User Privacy}{\UserPrivacy{1cm}}{10303\_UserPrivacy.pdf}{UserPrivacy}{Management + Governance}

\subsection{Migrate}
\gxs{Azure Migrate}{\AzrMigrate{1cm}}{10281\_AzureMigrate.pdf}{AzrMigrate}{Migrate}

\gxs{Data Box}{\DataBox{1cm}}{10094\_DataBox.pdf}{DataBox}{Migrate}

\gxs{Data Box Edge}{\DataBoxEdge{1cm}}{10095\_DataBoxEdge.pdf}{DataBoxEdge}{Migrate}

\gxs{Recovery Services Vaults}{\RecovSvcsVlts{1cm}}{00017\_RecoveryServicesVaults.pdf}{RecovSvcsVlts}{Migrate}

\subsection{Mixed Reality}
\gxs{Remote Rendering}{\RemoteRend{1cm}}{00698\_RemoteRendering.pdf}{RemoteRend}{Mixed Reality}

\subsection{Monitor}
\gxs{SAP Azure Monitor}{\SAPAzrMonitor{1cm}}{00438\_SAPAzureMonitor.pdf}{SAPAzrMonitor}{Monitor}

\subsection{Networking}
\gxs{Application Gateways}{\AppGateways{1cm}}{10076\_ApplicationGateways.pdf}{AppGateways}{Networking}

\gxs{Azure Firewall Manager}{\AzrFirewallMngr{1cm}}{00271\_AzureFirewallManager.pdf}{AzrFirewallMngr}{Networking}

\gxs{CDN Profiles}{\CDNProfs{1cm}}{00056\_CDNProfiles.pdf}{CDNProfs}{Networking}

\gxs{Connections}{\Connections{1cm}}{10081\_Connections.pdf}{Connections}{Networking}

\gxs{DDoS Protection Plans}{\DDoSProtectionPlans{1cm}}{10072\_DDoSProtectionPlans.pdf}{DDoSProtectionPlans}{Networking}

\gxs{DNS Zones}{\DNSZones{1cm}}{10064\_DNSZones.pdf}{DNSZones}{Networking}

\gxs{ExpressRoute Circuits}{\ExpressRouteCircuits{1cm}}{10079\_ExpressRouteCircuits.pdf}{ExpressRouteCircuits}{Networking}

\gxs{Firewalls}{\Firewalls{1cm}}{10084\_Firewalls.pdf}{Firewalls}{Networking}

\gxs{Front Doors}{\FrontDoors{1cm}}{10073\_FrontDoors.pdf}{FrontDoors}{Networking}

\gxs{IP Groups}{\IPGrp{1cm}}{00701\_IPGroups.pdf}{IPGrp}{Networking}

\gxs{Load Balancers}{\LBs{1cm}}{10062\_LoadBalancers.pdf}{LBs}{Networking}

\gxs{NAT}{\NAT{1cm}}{10310\_NAT.pdf}{NAT}{Networking}

\gxs{Network Interfaces}{\NetworkInterfaces{1cm}}{10080\_NetworkInterfaces.pdf}{NetworkInterfaces}{Networking}

\gxs{Network Security Groups}{\NetworkSecGrp{1cm}}{10067\_NetworkSecurityGroups.pdf}{NetworkSecGrp}{Networking}

\gxs{Network Watcher}{\NetworkWatcher{1cm}}{10066\_NetworkWatcher.pdf}{NetworkWatcher}{Networking}

\gxs{Private Link}{\PrivateLink{1cm}}{00427\_PrivateLink.pdf}{PrivateLink}{Networking}

\gxs{Private Link Service}{\PrivateLinkSvc{1cm}}{01105\_PrivateLinkService.pdf}{PrivateLinkSvc}{Networking}

\gxs{Proximity Placement Groups}{\ProximityPlacementGrp{1cm}}{10365\_ProximityPlacementGroups.pdf}{ProximityPlacementGrp}{Networking}

\gxs{Public IP Addresses}{\PublicIPAddresses{1cm}}{10069\_PublicIPAddresses.pdf}{PublicIPAddresses}{Networking}

\gxs{Public IP Addresses Classic}{\PublicIPAddressesClassic{1cm}}{10068\_PublicIPAddressesClassic.pdf}{PublicIPAddressesClassic}{Networking}

\gxs{Public IP Prefixes}{\PublicIPPrefixes{1cm}}{10372\_PublicIPPrefixes.pdf}{PublicIPPrefixes}{Networking}

\gxs{Reserved IP Addresses Classic}{\ReservedIPAddressesClassic{1cm}}{10371\_ReservedIPAddressesClassic.pdf}{ReservedIPAddressesClassic}{Networking}

\gxs{Route Filters}{\RouteFilters{1cm}}{10071\_RouteFilters.pdf}{RouteFilters}{Networking}

\gxs{Route Tables}{\RouteTables{1cm}}{10082\_RouteTables.pdf}{RouteTables}{Networking}

\gxs{Service Endpoint Policies}{\SvcEndpointPolicies{1cm}}{10085\_ServiceEndpointPolicies.pdf}{SvcEndpointPolicies}{Networking}

\gxs{Traffic Manager Profiles}{\TrafficMngrProfs{1cm}}{10065\_TrafficManagerProfiles.pdf}{TrafficMngrProfs}{Networking}

\gxs{Virtual Network Gateways}{\VirtNetworkGateways{1cm}}{10063\_VirtualNetworkGateways.pdf}{VirtNetworkGateways}{Networking}

\gxs{Virtual Networks}{\VirtNetworks{1cm}}{10061\_VirtualNetworks.pdf}{VirtNetworks}{Networking}

\gxs{Virtual Networks Classic}{\VirtNetworksClassic{1cm}}{10075\_VirtualNetworksClassic.pdf}{VirtNetworksClassic}{Networking}

\gxs{Virtual WANs}{\VirtWANs{1cm}}{10353\_VirtualWANs.pdf}{VirtWANs}{Networking}

\gxs{Web Application Firewall PoliciesWAF}{\WebAppFirewallPoliciesWAF{1cm}}{10362\_WebApplicationFirewallPoliciesWAF.pdf}{WebAppFirewallPoliciesWAF}{Networking}

\subsection{Other}
\gxs{Detonation}{\Detonation{1cm}}{00378\_Detonation.pdf}{Detonation}{Other}

\gxs{Instance Pools}{\InstPools{1cm}}{10139\_InstancePools.pdf}{InstPools}{Other}

\gxs{Internet Analyzer Profiles}{\INetAnalyzerProfs{1cm}}{00469\_InternetAnalyzerProfiles.pdf}{INetAnalyzerProfs}{Other}

\gxs{Peering Service}{\PeeringSvc{1cm}}{00970\_PeeringService.pdf}{PeeringSvc}{Other}

\gxs{Universal Print}{\UniverPrint{1cm}}{00571\_UniversalPrint.pdf}{UniverPrint}{Other}

\gxs{Windows Virtual Desktop}{\WinVirtDesktop{1cm}}{00327\_WindowsVirtualDesktop.pdf}{WinVirtDesktop}{Other}

\subsection{Preview}
\gxs{Azure Cloud Shell}{\AzrCloudShell{1cm}}{00559\_AzureCloudShell.pdf}{AzrCloudShell}{Preview}

\gxs{Azure Sphere}{\AzrSphere{1cm}}{10190\_AzureSphere.pdf}{AzrSphere}{Preview}

\gxs{Azure Workbooks}{\AzrWorkbooks{1cm}}{02189\_AzureWorkbooks.pdf}{AzrWorkbooks}{Preview}

\gxs{IoT Edge}{\IoTEdge{1cm}}{10186\_IoTEdge.pdf}{IoTEdge}{Preview}

\gxs{Private Link Hub}{\PrivateLinkHub{1cm}}{02209\_PrivateLinkHub.pdf}{PrivateLinkHub}{Preview}

\gxs{RTOS}{\RTOS{1cm}}{10778\_RTOS.pdf}{RTOS}{Preview}

\gxs{Static Apps}{\StaticApps{1cm}}{01007\_StaticApps.pdf}{StaticApps}{Preview}

\gxs{Time Series Data Sets}{\TSDataSets{1cm}}{10198\_TimeSeriesDataSets.pdf}{TSDataSets}{Preview}

\gxs{Web Environment}{\WebEnvironment{1cm}}{10846\_WebEnvironment.pdf}{WebEnvironment}{Preview}

\subsection{Security}
\gxs{Application Security Groups}{\AppSecGrp{1cm}}{10244\_ApplicationSecurityGroups.pdf}{AppSecGrp}{Security}

\gxs{Azure Sentinel}{\AzrSentinel{1cm}}{10248\_AzureSentinel.pdf}{AzrSentinel}{Security}

\gxs{Conditional Access}{\ConditionalAccess{1cm}}{10233\_ConditionalAccess.pdf}{ConditionalAccess}{Security}

\gxs{ExtendedSecurityUpdates}{\ExtendedSecUpdates{1cm}}{10572\_ExtendedSecurityUpdates.pdf}{ExtendedSecUpdates}{Security}

\gxs{Key Vaults}{\KeyVlts{1cm}}{10245\_KeyVaults.pdf}{KeyVlts}{Security}

\gxs{Security Center}{\SecCenter{1cm}}{10241\_SecurityCenter.pdf}{SecCenter}{Security}

\subsection{Storage}
\gxs{Azure HCP Cache}{\AzrHCPCache{1cm}}{00776\_AzureHCPCache.pdf}{AzrHCPCache}{Storage}

\gxs{Azure NetApp Files}{\AzrNetAppFiles{1cm}}{10096\_AzureNetAppFiles.pdf}{AzrNetAppFiles}{Storage}

\gxs{Azure Stack Edge}{\AzrStackEdge{1cm}}{00691\_AzureStackEdge.pdf}{AzrStackEdge}{Storage}

\gxs{Data Box}{\DataBox{1cm}}{10094\_DataBox.pdf}{DataBox}{Storage}

\gxs{Data Box Edge}{\DataBoxEdge{1cm}}{10095\_DataBoxEdge.pdf}{DataBoxEdge}{Storage}

\gxs{Data Lake Storage Gen1}{\DataLakeStorGenOne{1cm}}{10090\_DataLakeStorageGen1.pdf}{DataLakeStorGenOne}{Storage}

\gxs{Data Share Invitations}{\DataShareInvitations{1cm}}{10097\_DataShareInvitations.pdf}{DataShareInvitations}{Storage}

\gxs{Data Shares}{\DataShares{1cm}}{10098\_DataShares.pdf}{DataShares}{Storage}

\gxs{Import Export Jobs}{\ImportExportJobs{1cm}}{10100\_ImportExportJobs.pdf}{ImportExportJobs}{Storage}

\gxs{Recovery Services Vaults}{\RecovSvcsVlts{1cm}}{00017\_RecoveryServicesVaults.pdf}{RecovSvcsVlts}{Storage}

\gxs{StorSimple Data Managers}{\StorSimpleDataMngrs{1cm}}{10092\_StorSimpleDataManagers.pdf}{StorSimpleDataMngrs}{Storage}

\gxs{StorSimple Device Managers}{\StorSimpleDevMngrs{1cm}}{10089\_StorSimpleDeviceManagers.pdf}{StorSimpleDevMngrs}{Storage}

\gxs{Storage Accounts}{\StorAccnts{1cm}}{10086\_StorageAccounts.pdf}{StorAccnts}{Storage}

\gxs{Storage Accounts Classic}{\StorAccntsClassic{1cm}}{10087\_StorageAccountsClassic.pdf}{StorAccntsClassic}{Storage}

\gxs{Storage Sync Services}{\StorSyncSvcs{1cm}}{10093\_StorageSyncServices.pdf}{StorSyncSvcs}{Storage}

\subsection{Web}
\gxs{Azure Media Service}{\AzrMediaSvc{1cm}}{10309\_AzureMediaService.pdf}{AzrMediaSvc}{Web}

\gxs{Notification Hub Namespaces}{\NotifHubNS{1cm}}{10053\_NotificationHubNamespaces.pdf}{NotifHubNS}{Web}


\end{multicols}\clearpage

%%%%%%%%%%%%% INDEXES
\normalsize

\appendix
\section{Mapping of icon names to macros names}
In order to have reasonably short macro names, the following mappings have been made by \texttt{aws2tex}, being careful not to introduce any naming collisions.
In addition, \LaTeX\ macros cannot contain numbers, so all numbers have been mapped to words, i.e. 1 $\rightarrow$ One, 5 $\rightarrow$ Five, etc.
The below mappings are applied in order, followed by the number mappings.
{\tiny
\begin{multicols}{3}
  \begin{enumerate}
  \item example
  \end{enumerate}
\end{multicols}
}

\printindex
\printindex[macros]

\end{document}
